\documentclass[]{article}
\RequirePackage{amsmath}

\usepackage{graphicx}
%\graphicspath{{./figures/}}
\usepackage{amssymb}
\usepackage{color}
\usepackage{hyperref}
\usepackage{algorithm}
\usepackage{algpseudocode}
\bibliographystyle{IEEEtran}
\usepackage{amsthm}

\newcommand{\knote}[1]{\textcolor{green}{A: {#1}}}
\newcommand{\dnote}[1]{\textcolor{red}{D: {#1}}}
\newcommand{\vk}[1]{\textcolor{blue}{V: {#1}}}
\newcommand{\lnote}[1]{\textcolor{cyan}{L: {#1}}}

\newcommand{\Ticket}{\mathsf{ticket}}
\newcommand{\KeyGen}{\mathsf{KeyGen}}
\newcommand{\PuzzleGen}{\mathsf{PuzzleGen}}
\newcommand{\Eval}{\mathsf{Evaluate}}
\newcommand{\Verify}{\mathsf{Verify}}
\newcommand{\ObtainTicket}{\mathsf{ObtainTicket}}
\newcommand{\Name}{$Autolykos$}
\def\Let#1#2{\State #1 $:=$ #2}
\def\LetRnd#1#2{\State #1 $\gets$ #2}

\newcommand{\pk}{\mathsf{pk}}
\newcommand{\sk}{\mathsf{sk}}

\newtheorem{theorem}{Theorem}
\newtheorem{definition}{Definition}
\newtheorem{claim}{Claim}
\newtheorem{game}{Game}

\begin{document}
    \title{\Name: The Ergo Platform PoW Puzzle}

    \author{Alexander Chepurnoy, Vasily Kharin, Dmitry Meshkov}

    \date{\today}
    \maketitle

    %    \begin{abstract}
    %        This document contains the full description of \Name~-- the PoW protocol that is going to be used in Ergo platform.
    %    \end{abstract}


    \section{Introduction}

    Security of Proof-of-Work blockchains relies 
    on multiple miners trying to produce new blocks by
    participating in PoW puzzle lottery, and the network is secure if the
    majority of them are honest. However, the reality becomes much more complicated
    than the original one-CPU-one-vote idea from the Bitcoin whitepaper\cite{nakamoto2008bitcoin}.

    The first threat to decentralization came from mining pools -- miners tend
    to unite in mining pools.
    Regardless of the PoW algorithm number of pools controlling more then 50\% of
    computational power is usually quite small: 4 pools in Bitcoin, 2 in Ethereum, 3 in ZCash, etc.
    This problem led to the notion of non-outsourceable puzzles~\cite{miller2015nonoutsourceable,daian2017piecework}.
    These are the puzzles constructed in such a way that if a mining pool outsources the puzzle
    to a miner, miner can recover pool's private key and steal the reward with a non-negligible probability.
    However the existing solutions either have too large solution size (kilobyte is already
    on the edge of acceptability for distributed ledgers) or very specific and
    can not be modified or extended in any way without breaking non-outsourceability.

    The second threat to cryptocurrencies decentralization is that ASIC-equipped miners are
    able to find PoW solutions orders of magnitude faster and more efficiently
    than miners equipped with the commodity hardware. In order to reduce the
    disparity between the ASICs and regular hardware, memory-bound computations
    where proposed in~\cite{dwork2003memory}. The most interesting practical
    examples are two
    asymmetric memory-hard PoW schemes which require significantly less memory
    to verify a solution than to find it~\cite{biryukov2017equihash,ethHash}.
    Despite the fact that ASICs already exist for both of them~\cite{ETHAsics,EquihashAsics},
    they remain the only asymmetric memory-hard PoW algorithms in use.

    In this paper we propose \Name{} --- new asymmetric memory-hard non-outsourceable PoW puzzle.
    In Section~\ref{puzzle} we provide a full
    specification of \Name, while in Section~\ref{discussion} we discuss its
    properties. Few auxiliary algorithms are placed in~\nameref{appendix}.

    \section{Non-outsourceability of Keyed Puzzles}
    Proof-of-Work puzzles are usually constructed as sequence of independent
    tries for tickets in order to find one which satisfies
    \begin{equation}
        \Eval(m,\mathsf{ticket})<z\,,
        \label{eq:default_pow}
    \end{equation}
    where $\Eval$ is some function (such as hash in Bitcoin), $m$ is the payload,
    and the space of valid tickets can also be a subject to some constraints.
    Miller's paper exploits the idea that the puzzle can be constructed in such
    a way that the worker who was capable of constructing one winning ticket can
    also produce another one inicitaing the race with the outsourcer. In what
    follows we reason in a slightly diffiernt manner: we embed the public key in
    the puzzle instance such that the checks~\eqref{eq:default_pow} can only be
    performed efficiently with the help of the secret key. Therefore,
    either every ticket try requires additional communication between outsourcer
    and worker to check condition~\eqref{eq:default_pow}, or outsourcer provides
    the secret key to the worker putting itself at risk of rewards theft.

    The simplest way to implement this requirement is to use signature instead
    of $\Eval$. For example, the second field of Schnorr signature (response in
    corresponding $\Sigma$-protocol) would suffice. However, in this case one is
    restricted to a narrow set of admissible puzzles: the freedom of choice is
    only in discrete logarithm group and hash function in use. In order to
    broaden the class of puzzles one can also embed the secret key in ticket
    generation routine (to make winning ticket generation more efficient with
    the use of secret key). This can be used to convert puzzles more sophisticated
    than simple trial of pseudorandom numbers to non-outsourceable form. In what
    follows $\lambda$ stands for security parameter.
    \begin{definition}
        {\bf Keyed Proof-of-Work Puzzle} is parametrized by payload $m$ and
        current target value $z$. It consists of the following algorithms
        \begin{itemize}
            \item $\KeyGen(1^\lambda)\rightarrow (pk,sk)$ is a randomized
                algorithm generating a pair (public key, secret key)
            \item $\PuzzleGen(pk)\rightarrow \mathsf{puz}$ is a deterministic
                algorithm generating a puzzle instance
            \item
                $\ObtainTicket(\mathsf{puz},sk,m,z)\rightarrow\Ticket \,
                \mathrm{or}\,\bot$
                is a randomized algorithm generating ticket of 
                $\mathsf{puz}$ solution (not necessarily satisfying target
                criterium)
            \item $\Eval(\Ticket,sk,m)\rightarrow d$
                returns a number. The ticket is winning iff $d<z$, where $z$
                is the current target value
            \item $\Verify(\Ticket,pk,m,d)$ returns $\mathsf{True}$ iff
                $\Ticket$ is a valid ticket for $\mathsf{puz}$ with payload $m$,
                and $\Eval(\Ticket,sk,m)=d$. It returns $\mathsf{False}$
                otherwise
        \end{itemize}
    \end{definition}
    Tuple $(\Ticket, pk, m, d)$ can be open public, and is treated as the
    solution. The signature corresponding to $pk$ is required to claim the block
    reward. The network checks corretness using $\Verify$ function and matching
    target condition $d<z$.

    Mining consists in repeated calls of $\ObtainTicket$ function.
    The $\Eval$ function can be implemented inside (then $\ObtainTicket$
    returns winning tickets only) or have separate calls.

    \begin{definition}
        A keyed Proof-of-Work puzzle $\mathcal{P}$ is {\bf non-transferrable} if
        \vk{No gain in generating ticket for new pair $(m',pk')\ne (m_i,pk_i)$
            for some $i$ given set $\{(\Ticket_j, pk_j, m_j, d_j)\}$.}
    \end{definition}

    \begin{definition}
        A keyed Proof-of-Work puzzle $\mathcal{P}$ is {\bf secure under key
        extraction} if for any efficient adversary $\mathcal{A}$
        \begin{equation}
            \mathsf{Pr}[\mathcal{A}(\Ticket,pk,m,d)=sk]\le\mathsf{negl}(\lambda).
        \end{equation}
    \end{definition}

    \begin{definition}
        We call a keyed Proof-of-Work puzzle $\mathcal{P}$ {\bf
        non-outsourceable} if it is secure under key extraction,
        non-transferrable, and a pair of functions $\Eval$, $\Verify$
        constitutes non-interactive proof of knowledge system for $sk$.
    \end{definition}

    That is, a keyed PoW puzzle is non-outsourceable if previous solutions to
    the puzzle do not affect the performance of future solving, the secret key
    cannot be extracted from the solution, and the worker cannot check
    whether its solution matches the target without knowing the secret key.

    \section{Ergo PoW puzzle}
    \label{puzzle}

    The proposed scheme requires following components:
    \begin{enumerate}
        \item Cyclic group $\mathbb{G}$ of prime order~$q$ with fixed generator~$g$
        and identity element~$e$.
        Curve 25519 group is used for this purposes.
        \item Number of elements $k$ required in the solution. Value $k=32$ is used in
            implementation.
        \item Number $N$ of elements in the list
            $R\subset\mathbb{Z}/q\mathbb{Z}$ to be stored in miner's memory.
            Value $N=2^{26}$ is used in implementation.
        \item Hash function $H$ which returns the values in $\mathbb{Z}/q\mathbb{Z}$.
        Particular implementation is based on Blake2b256 and is described in Alg.\ref{alg:H}.
        \item Hash function $genIndexes$ which returns a list of numbers from
            $0\dots(N-1)$ of size $k$.
        It is based on Blake2b256 and is described in Alg.\ref{alg:genIndexes}.
        \item Target interval parameter $b$, that is recalculated via difficulty adjustment rules.
        \item Constant message $M=[0,\dots,1023].flatMap(i => Longs.toByteArray(i))$ that is used to enlarge message size and increase elements calculation time.
    \end{enumerate}

    \Name{} is based on one list $k$-sum problem: miner should find
    $k$ elements from the pre-defined list $R$ of size $N$, such that
    $\sum_{j \in J} r_{j} - sk = d$ is in the interval $\{-b,\dots,0,\dots,b\mod q\}$.
    In addition, we require set of element indexes $J$ to be obtained
    by one-way pseudo-random function $genIndexes$. This prevents optimizations as
    soon as it is hard to find such a seed,
    that $genIndexes(seed)$ returns the desired indexes.

    Thus we assume that the only option for miner is to use the simple brute-force algorithm~\ref{alg:prove} to
    create a valid block.

    \begin{algorithm}[H]
        \caption{Block mining}
        \label{alg:prove}
        \begin{algorithmic}[1]
            \State \textbf{Input}: latest block header $hdr$, key pair $pk=g^{sk}$
            \State Generate randomly a new key pair $w=g^x$
            \Let{$m$}{$Blake2b256(hdr.bytesWithoutPow)$}
            \While{$true$}
            \LetRnd{$nonce$}{$\mathsf{rand}$}
            \Let{$J$}{$genIndexes(m||nonce)$}
            \Let{$d$}{$\sum_{j \in J}{H(j||M||pk||m||w)} \cdot x - sk \mod q$}
            \If{$d \le b$}
            \State \Return $(m,pk,w,nonce,d)$
            \EndIf
            \EndWhile
        \end{algorithmic}
    \end{algorithm}

    Note that although the mining process utilizes private keys, solution itself
    only contains public keys. Solution verification can be performed by Alg.~\ref{alg:verify}.

    \begin{algorithm}[H]
        \caption{Solution verification}
        \label{alg:verify}
        \begin{algorithmic}[1]
            \State \textbf{Input}: $m,pk,w,nonce,d$
            \State require $d\in\{-b,\dots,0,\dots, b\mod q\}$
            \State require $pk,w\in \mathbb{G}$ and $pk,w \ne e$
            \Let{$J$}{$genIndexes(m||nonce)$}
            \Let{$f$}{$\sum_{j \in J} H(j||M||pk||m||w)$}
            \State require $w^f = g^dpk$
        \end{algorithmic}
    \end{algorithm}

    \section{Security and Non-Outsourceability}
    \begin{claim}
        ErgoPoW is a keyed PoW puzzle.
    \end{claim}
    First, notice that in Algorithm~\ref{alg:prove} we refer to construction
    $f(m,nonce,w,pk)=\sum_{j\in genIndexes(m||nonce)} H(j||M||pk||w)$ as a hash
    function. Public key plays a role of commitment. Therefore, the pair
    $(pk,d)$ is a Schnorr signature with a public key $w$ over the message
    $(m,nonce)$ with a hash function $f$. If one denotes $e$ the corresponding
    value of $f$, and pass to more common notations:
    $e=f(m,nonce,w,w^eg^{-d})$. The puzzle consists in trying
    different nonces and keys in order for signature to satisfy
    $d\in\{-b,\dots,0,\dots,b\}$. 
    \begin{claim}
        ErgoPoW is non-transferrable
    \end{claim}
    
    \begin{claim}
        ErgoPoW is secure under key extraction
    \end{claim}
    \begin{theorem}
        ErgoPoW is non-outsourceable as a keyed PoW puzzle
    \end{theorem}
    %Security follows from the security of Schnor
    %signatures, and outsourcing the puzzle is equivalent to outsourcing the
    %signature (or parts of signature creation routine). The only difference from
    %conventional setup is the design of function $f$. It must be constructed in such
    %a way that efficient massive evaluations with different nonces require
    %allocating large amount of memory (benefitting from data reuse), whereas
    %single evaluation on verifier's side can be done ``on fly''.

    %\vk{I would exchange $pk$ and $w$, s.t. $pk$ is exactly public key and $w$
    %is commitment}

    \section{Discussion}
    \label{discussion}

    \Name{} mining~\ref{alg:prove} and verification~\ref{alg:verify} algorithms
    are similar to Schnorr signature algorithm. Therefore, efficient solving
    requires access to private key $sk$. This fact implies that in order to get
    complete solution work outsourcer needs to pass the secret key to the miner,
    putting itself at risk of loosing block rewards.
    This fact discourages formation of centralized pools and centralization of the network in the hands of pool operators.

    Mining algorithm~\ref{alg:prove} can be optimized, if miner pre-calculates
    hashes\newline $\{H(j||M||pk||m||w)|i \in [0,N)\}$ and stores them in the memory.
    Every pre-calculated hash occupies 32 bytes, so the whole list of $N$ elements
    occupies $N \cdot 32 = 2 Gb$, making this optimization memory-hard.
    However, the miner profit will also be significant: assuming GPU hashrate
    $G = 2^{30} H/s$~\cite{gpuHashrate} and block interval $t=120~s$, every element
    will be used $(G / N) \cdot k \cdot t = 3 \cdot 10^4$ times on average.
    Thus, if a miner tries to 
    recalculate hashes ``on the fly'' instead of storing them, the number of
    calls of $H$ grows $10^4$ times, which is rather significant value.

    Protocol is quite efficient in terms of size: in addition to payload, block header should
    contain as few as 2 public keys of size 32 bytes, number $d$ that is at most 32 bytes
    (but contains a lot of leading zeros in case of small target $b$) and an
    8-bytes long nonce. Header verification requires verifier to calculate 1 $genIndexes$
    hash, $k$ hashes $H$ and perform two exponentiations in the group. Reference
    Scala implementation~\cite{ergoGit} allows to verify block header in 2
    milliseconds on Intel Core i5-7200U, 2.5GHz.

    \bibliography{references}

    \section*{Appendix}
    \label{appendix}

    Implementation of hash function $H$ which returns the values in $\mathbb{Z}/q\mathbb{Z}$:

    \begin{algorithm}[H]
        \caption{Numeric hash}
        \label{alg:H}
        \begin{algorithmic}[1]
            \Function{H}{$input$}
            \Let{$validRange$}{$(2^{256} / q) \cdot q$}
            \Let{$hashed$}{$Blake2b256(input)$}
            \If{$hashed < validRange$}
            \State \Return $hashed.mod(q)$
            \Else
            \State \Return $H(hashed)$
            \EndIf
            \EndFunction
        \end{algorithmic}
    \end{algorithm}

    Implementation of hash function $genIndexes$ which returns a list of size $k$ with numbers in $0\dots (N-1)$:

    \begin{algorithm}[H]
        \caption{Index generator}
        \label{alg:genIndexes}
        \begin{algorithmic}[1]
            \Function{genIndexes}{$seed$}
            \Let{$hash$}{$Blake2b256(seed)$}
            \Let{$extendedHash$}{$hash||hash$}
            \State \Return $(0\dots{k-1}).map(i => extendedHash.slice(i,i+4).mod(N))$
            \EndFunction
        \end{algorithmic}
    \end{algorithm}

\end{document}
